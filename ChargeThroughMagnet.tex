\documentclass[letterpaper,11pt,pdftex]{article}
\usepackage[left=1in,right=1in,bottom=1in]{geometry}

%\usepackage{mathptmx}
\usepackage[T1]{fontenc}
\usepackage{textcomp}

\usepackage{listings}
\usepackage{amsmath}
\usepackage{amssymb}
\usepackage{upgreek}
\usepackage[load-configurations=abbreviations,detect-all=true]{siunitx}
\usepackage[version=3]{mhchem}
\usepackage[all]{xy}
\usepackage{url}

\usepackage{datetime}
%\renewcommand{\dateseparator}{}
\newdateformat{yyyymmdd}{\twodigit\THEYEAR\twodigit\THEMONTH\twodigit\THEDAY}

\usepackage{graphicx}
\usepackage{lscape}
%\usepackage{indentfirst}
\pagestyle{plain}
\usepackage{setspace}
%\doublespacing
\renewcommand{\baselinestretch}{1.1}
\usepackage[numbers,sort&compress]{natbib}
%\input{defs}
\newcommand*\chem[1]{\ensuremath{\mathrm{#1}}}
\usepackage{float}
\usepackage{graphics}

\title{Calculate the magnetic field of bending magnet }
\author{Zhen Wang}
\date{\today\footnote{Created: 20141103}}
%\date{\yyyymmdd\today}
\begin{document}
\maketitle

% \section{magnetic field}
According to 
\begin{align}
    \label{eq:1}
    Bqv &= \frac{m{v}^{2}}{r} \\ 
    {E}_{k} &= \frac{1}{2}m{v}^{2 }= Uq             
\end{align}
we can deduce
\begin{align}
 B = \left\{
            \begin{array}{l l}
                \frac{\sqrt{2}}{r}\sqrt{\frac{m}{q}}\sqrt{U} & \quad $q > 0$ \\ 
                -\frac{\sqrt{2}}{r}\sqrt{-\frac{m}{q}}\sqrt{-U} & \quad $q < 0$ 
            \end{array}
        \right.
\end{align}        
If units are given:
\begin{align}
    U &: [\si{kV}],  & \SI{1}{kV}  &= \SI{e3}{V} \\ 
    m &: [\si{amu}], & \SI{1}{amu} &= \SI{1.67e-27}{kg} \\ 
    q &: [\si{e}],   & \SI{1}{e}   &= \SI{1.6e-19}{C} \\ 
    r &: [\si{cm}],  & \SI{1}{cm}  &= \SI{e-2}{m}
\end{align}
\begingroup
\Large
\begin{align}
    B & = \left\{
            \begin{array}{l l}
                \frac{\sqrt{2}}{r[\si{cm}]\SI{e-2}{}}\sqrt{\frac{m[\si{amu}]}{q[\si{e}]}}\sqrt{\frac{\SI{1.67e-27}{}}{\SI{1.6e-19}{}}}\sqrt{U[\si{kV}]}10^{\frac{3}{2}}  & \quad $q > 0$  \\ 
                -\frac{\sqrt{2}}{r[\si{cm}]\SI{e-2}{}}\sqrt{-\frac{m[\si{amu}]}{q[\si{e}]}}\sqrt{\frac{\SI{1.67e-27}{}}{\SI{1.6e-19}{}}}\sqrt{-U[\si{kV}]}10^{\frac{3}{2}}  & \quad $q < 0$  
            \end{array}
        \right. \\   
      & = \left\{
            \begin{array}{l l}
                \frac{0.457}{r[\si{cm}]}\sqrt{\frac{m[\si{amu}]}{q[\si{e}]}}\sqrt{U[\si{kV}]} [\si{T}] & \quad $q > 0$  \\ 
                -\frac{0.457}{r[\si{cm}]}\sqrt{-\frac{m[\si{amu}]}{q[\si{e}]}}\sqrt{-U[\si{kV}]} [\si{T}] & \quad $q < 0$ 
            \end{array}
        \right.       
\end{align}
\endgroup
\setlength{\tabcolsep}{9pt} % space between columns
\renewcommand{\arraystretch}{1.5}% space between rows

\begin{table}[H]
    \centering
    \resizebox{0.75\columnwidth}{!}{
        \begin{tabular}{| l | l | l | l| l |}
        \hline
        Ion & \(\displaystyle \frac{m}{q}\) & r[\si{cm}] & U[\si{kV}] & B[\si{T}] \\ \hline
        \chem{H_{2}^{2+}} & 1 & 57 & 7 & 0.021 \\ \hline
        \chem{H_{2}^{+}} & 2 & 57 & 7 & 0.030 \\ \hline
        \chem{O^{2-}} & -8 & 57 & -7 & -0.060 \\ \hline
        \chem{Ne^{+}} & 20 & 57 & 7 & 0.095\\ \hline
        \chem{N_{2}^{+}} & 28 & 57 & 7 & 0.112\\ \hline
        \chem{O_{2}^{+}} & 32 & 57 & 7 & 0.120\\ \hline
        \chem{Ar^{+}} & 40 & 57 & 7 & 0.134\\ \hline
      \end{tabular}
  }
\end{table}

\renewcommand{\bibname}{Bibliography}
\bibliographystyle{plainnat}
% \bibliography{refs}
% \begin{thebibliography}{99}
% \bibitem{tag0} Bib1 \url{http://www.google.com}
% \end{thebibliography}
\end{document}

%%% Local Variables: 
%%% mode: latex
%%% TeX-engine: default
%%% TeX-PDF-mode: t
%%% TeX-master: t
%%% fill-column: 97
%%% End: 

