\documentclass[letterpaper,11pt,pdftex]{article}
\usepackage[left=1in,right=1in,bottom=1in]{geometry}

%\usepackage{mathptmx}
\usepackage[T1]{fontenc}
\usepackage{textcomp}

\usepackage{listings}
\usepackage{amsmath}
\usepackage{amssymb}
\usepackage{upgreek}
\usepackage[load-configurations=abbreviations,detect-all=true]{siunitx}
\usepackage[version=3]{mhchem}
\usepackage[all]{xy}
\usepackage{url}

\usepackage{datetime}
%\renewcommand{\dateseparator}{}
\newdateformat{yyyymmdd}{\twodigit\THEYEAR\twodigit\THEMONTH\twodigit\THEDAY}

\usepackage{graphicx}
\usepackage{lscape}
%\usepackage{indentfirst}
\pagestyle{plain}
\usepackage{setspace}
%\doublespacing
\renewcommand{\baselinestretch}{1.1}
\usepackage[numbers,sort&compress]{natbib}

%\input{defs}

\title{Calculate the magnetic field of bending magnet }
\author{Zhen Wang}
\date{\today\footnote{Created: 20141103}}
%\date{\yyyymmdd\today}
\begin{document}
\maketitle

\section{Introduction}
\begin{equation}
    \label{simple_equation}
    \frac{m{v}^{2}}{r} = \frac{Bqv}{r}
\end{equation}
\begin{equation}
    \label{simple_equation}
    \frac{1}{2}m{v}^{2} = {E}_{k}
\end{equation}
Combined with the two equations, 
\begin{equation}
    \label{simple_equation}
    B = \frac{\sqrt{2m{E}_{k}}}{rq}
\end{equation}
could be deduced.\\ 
If units are given:
$${E}_{k} : \si{keV},       \SI{1}{eV} = \SI{1.6e-19}{J}$$;
$$m : \si{amu},  1 amu = 1.67 x 10^{-27} kg;$$ 
$$q : {e}^{-}, 1 e^{-} = 1.6x 10^{-19} C;$$
$$r : cm.  $$

The function $B = \frac{\sqrt{2m x 1.67 x 10^{-27} x {E}_{k} x 1.6x 10^{-19} x 10^{3} }}{r x 10^{-2} x q x 1.6x 10^{-19} } = 0.457 \frac{\sqrt{m{E}_{k}}}{qr}$
$$\frac{m{v}^{2}}{r} = \frac{Bqv}{r}$$
$$\frac{1}{2}m{v}^{2} = {E}_{k}$$
$$B = \frac{\sqrt{2m{E}_{k}}}{rq}$$
\begin{equation}
    \label{simple_equation}
    \alpha = \sqrt{ \beta }
\end{equation}

\subsection{Subsection Heading Here}
Write your subsection text here.



\section{Conclusion}
Write your conclusion here.

\renewcommand{\bibname}{Bibliography}
\bibliographystyle{plainnat}
% \bibliography{refs}
% \begin{thebibliography}{99}
% \bibitem{tag0} Bib1 \url{http://www.google.com}
% \end{thebibliography}
\end{document}

%%% Local Variables: 
%%% mode: latex
%%% TeX-engine: default
%%% TeX-PDF-mode: t
%%% TeX-master: t
%%% fill-column: 97
%%% End: 
